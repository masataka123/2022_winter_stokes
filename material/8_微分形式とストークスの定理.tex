\documentclass[dvipdfmx,a4paper,11pt]{article}
\usepackage[utf8]{inputenc}
%\usepackage[dvipdfmx]{hyperref} %リンクを有効にする
\usepackage{url} %同上
\usepackage{amsmath,amssymb} %もちろん
\usepackage{amsfonts,amsthm,mathtools} %もちろん
\usepackage{braket,physics} %あると便利なやつ
\usepackage{bm} %ラプラシアンで使った
\usepackage[top=25truemm,bottom=25truemm,left=25truemm,right=25truemm]{geometry} %余白設定
\usepackage{latexsym} %ごくたまに必要になる
\renewcommand{\kanjifamilydefault}{\gtdefault}
\usepackage{otf} %宗教上の理由でmin10が嫌いなので


\usepackage[all]{xy}
\usepackage{amsthm,amsmath,amssymb,comment}
\usepackage{amsmath}    % \UTF{00E6}\UTF{0095}°\UTF{00E5}\UTF{00AD}\UTF{00A6}\UTF{00E7}\UTF{0094}¨
\usepackage{amssymb}  
\usepackage{color}
\usepackage{amscd}
\usepackage{amsthm}  
\usepackage{wrapfig}
\usepackage{comment}	
\usepackage{graphicx}
\usepackage{setspace}
\usepackage{pxrubrica}
\usepackage{enumitem}
\usepackage{mathrsfs} 
\usepackage[dvipdfmx]{hyperref}
\usepackage{mdwlist}
\usepackage{caption}


\setstretch{1.1}


\newcommand{\R}{\mathbb{R}}
\newcommand{\Z}{\mathbb{Z}}
\newcommand{\Q}{\mathbb{Q}} 
\newcommand{\N}{\mathbb{N}}
\newcommand{\C}{\mathbb{C}} 
\newcommand{\Sin}{\text{Sin}^{-1}} 
\newcommand{\Cos}{\text{Cos}^{-1}} 
\newcommand{\Tan}{\text{Tan}^{-1}} 
\newcommand{\invsin}{\text{Sin}^{-1}} 
\newcommand{\invcos}{\text{Cos}^{-1}} 
\newcommand{\invtan}{\text{Tan}^{-1}} 
\newcommand{\Area}{\text{Area}}
\newcommand{\vol}{\text{Vol}}
\newcommand{\maru}[1]{\raise0.2ex\hbox{\textcircled{\tiny{#1}}}}
\newcommand{\sgn}{{\rm sgn}}
\newcommand{\id}{{\rm id}}
\newcommand{\Sym}{{\rm Sym}}
\newcommand{\Supp}{{\rm Supp}}

%\newcommand{\rank}{{\rm rank}}



   %当然のようにやる.
\allowdisplaybreaks[4]
   %もちろん.
%\title{第1回. 多変数の連続写像 (岩井雅崇, 2020/10/06)}
%\author{岩井雅崇}
%\date{2020/10/06}
%ここまで今回の記事関係ない
\usepackage{tcolorbox}
\tcbuselibrary{breakable, skins, theorems}

\theoremstyle{definition}
\newtheorem{thm}{定理}
\newtheorem{lem}[thm]{補題}
\newtheorem{prop}[thm]{命題}
\newtheorem{cor}[thm]{系}
\newtheorem{claim}[thm]{主張}
\newtheorem{dfn}[thm]{定義}
\newtheorem{dfnthm}[thm]{定義と定理}
\newtheorem{rem}[thm]{補足}
\newtheorem{exa}[thm]{例}
\newtheorem{conj}[thm]{予想}
\newtheorem{prob}[thm]{問題}
\newtheorem{rema}[thm]{注意}

\DeclareMathOperator{\Ric}{Ric}
\DeclareMathOperator{\Vol}{Vol}
 \newcommand{\pdrv}[2]{\frac{\partial #1}{\partial #2}}
 \newcommand{\drv}[2]{\frac{d #1}{d#2}}
  \newcommand{\ppdrv}[3]{\frac{\partial #1}{\partial #2 \partial #3}}


%ここから本文.
\begin{document}
%\maketitle


\begin{center}
{\Large 8-12. 微分形式とストークスの定理}
\end{center}
\begin{flushright}
 岩井雅崇 2022/12/02
\end{flushright}

\section{ベクトル空間のテンソル積}

%以下断りがなければ$M$を$m$次元$C^\infty$級多様体とする.
    \begin{tcolorbox}[
    colback = white,
    colframe = green!35!black,
    fonttitle = \bfseries,
    breakable = true]
\begin{dfn}
$V$を$m$次元の$\R$ベクトル空間とする.
\begin{itemize}
 \setlength{\parskip}{0cm}
  \setlength{\itemsep}{2pt} 
\item $V$の\underline{双対ベクトル空間} $V^{*}$を
$
V^{*} := \{\omega : V \rightarrow \R  \,|\, \text{$\omega$は線型写像} \}
\text{とする.}
$
%と定義する.
\item $\{ e_1, \ldots, e_m\}$を$V$の基底とするとき, $1 \le i \le m$なる$i$について$\omega_{i} \in V^{*}$を
     $$
      \begin{matrix}
     \omega_{i} : &V & \rightarrow &\R\\
      &a_1e_1 + \cdots +a_m e_m& \mapsto& a_i
       \end{matrix}
      $$
 と定義する. $\{ \omega_1, \ldots, \omega_m\}$は$V^{*}$の基底で, $\{ e_1, \ldots, e_m\}$の双対基底と呼ばれる.
 \item $V$上の\underline{$k$次多重線型形式}とは$\omega : V \times \cdots \times V \rightarrow \R$となる写像で$\omega(v_1, \ldots, v_k)$が各$X_i$について線型であることとする. $V$上の$k$次多重線型形式なす$m^k$次元のベクトル空間を$\otimes^{k} V^{*}$と表す.
   \item $\omega \in \otimes^{k} V^{*}$が\underline{$k$次対称形式}であるとは, 任意の$k$次の置換$\sigma$と, 任意の$\omega(v_1, \ldots, v_k) \in V^{k}$について$\omega(v_{\sigma(1)}, \ldots, v_{\sigma(k)})=\omega(v_1, \ldots, v_k)$となることとする.
  \item $\omega \in \otimes^{k} V^{*}$が\underline{$k$次交代形式}であるとは, 任意の$k$次の置換$\sigma$と, 任意の$\omega(v_1, \ldots, v_k) \in V^{k}$について$\omega(v_{\sigma(1)}, \ldots, v_{\sigma(k)})=\sgn(\sigma)\omega(v_1, \ldots, v_k)$となることとする. $V$の$k$次交代形式の${}_m C_{k}$次元のベクトル空間を$\wedge^{k} V^{*}$で表す. 
\end{itemize}

    \end{dfn}
    \end{tcolorbox}
    \begin{exa}
 $\eta_1, \ldots, \eta_k \in V^{*}$について
     $$
      \begin{matrix}
     \eta_1\otimes \cdots \otimes \eta_k  : &V \times \cdots \times V & \rightarrow &\R &
     \eta_1\wedge \cdots \wedge \eta_k  : &V \times \cdots \times V & \rightarrow &\R \\
      &(v_1, \ldots, v_k)& \mapsto& \eta_1(v_1)\cdots \eta_{k}(v_k) & &(v_1, \ldots, v_k)& \mapsto& \det((\eta_i(v_j))_{1 \le i,j \le k})
       \end{matrix}
      $$
    と定義する. $\eta_1\otimes \cdots \otimes \eta_k \in \otimes^{k} V^{*}$であり\underline{$\eta_1, \ldots, \eta_k$のテンソル積}と呼ばれる. 

$\{ e_1, \ldots, e_m\}$を$V$の基底とし, $\{ \omega_1, \ldots, \omega_m\}$を$\{ e_1, \ldots, e_m\}$の双対基底とするとき, $\{ \omega_{i_1}\otimes \cdots \otimes  \omega_{i_k}\}_{i_1, \ldots, i_k=1, \ldots,m}$は$\otimes^{k} V^{*}$の基底となる.また$\{ \omega_{i_1}\wedge \cdots \wedge \omega_{i_k}\}_{1 \le i_1< \cdots < i_k\le m}$は$\wedge^{k} V^{*}$の基底となる.
定義から$\omega_2 \wedge \omega_1 = -\omega_1 \wedge \omega_2$や$\omega_1 \wedge \omega_1 = 0$であることがわかる.
\end{exa}
    
\section{微分形式}
    \begin{tcolorbox}[
    colback = white,
    colframe = green!35!black,
    fonttitle = \bfseries,
    breakable = true]
\begin{dfn}

\begin{itemize}
 \setlength{\parskip}{0cm}
  \setlength{\itemsep}{2pt} 
\item $p \in M$について, 接ベクトル空間$T_{p}M$の双対空間を\underline{余接ベクトル空間}と呼び$T_{p}^{*}M$と表す.
\item 任意の$p \in M$について$\omega_{p} \in T_{p}^{*}M$が一つずつ対応しているとき, その対応$\omega = \{ \omega_p\}_{p \in M}$を\underline{$M$上の1次微分形式}という.
\item 座標近傍$(U, x_1, \ldots, x_m)$について$(dx_{i})_{p}$を
    $$
     \begin{matrix}
    (dx_i )_{p} : &T_{p}M & \rightarrow &\R\\
    & a_1\left( \pdrv{}{x_1}\right)_p  + \cdots + a_m\left(\pdrv{}{x_m}\right)_p & \mapsto  &a_i
    \end{matrix}
     $$
    とし, $U$上の1次微分形式$dx_i  := \{ (dx_{i})_p\}_{p \in U}$と定義する. これにより$M$上の1次微分形式は座標近傍$(U, x_1, \ldots, x_m)$について, ある$U$上の関数$f_i : U \rightarrow \R$があって
    $$
    \omega|_{U} = f_1dx_1 + \cdots + f_mdx_m
    $$
    とかける. 各座標近傍$(U, x_1, \ldots, x_m)$について上の$f_i $が$C^{\infty}$級となるとき, $\omega$は\underline{$C^{\infty}$級1次微分形式}という.
    
    %$M$上の$C^{\infty}$級ベクトル場の集合を$\mathscr{X}(M)$で表す. 
\end{itemize}
    \end{dfn}
    \end{tcolorbox}
    
   \begin{exa}
$f : M\rightarrow \R$を$C^{\infty}$級写像とすると, 微分写像$df_{p} : T_{p}M \rightarrow T_{f(p)}\R \cong \R$により, $df:= \{df_{p}\}_{p \in M}$は$M$上の1次微分形式だと思える.
 座標近傍$(U, x_1, \ldots, x_m)$を用いて1次微分形式$df$は
 $$
 df|_{U} = \pdrv{f}{x_1} dx_1 + \cdots +\pdrv{f}{x_n}dx_{n}
 \text{と表せられる.}
 $$
 %と表すことができる. 
\end{exa}
    
     \begin{tcolorbox}[
    colback = white,
    colframe = green!35!black,
    fonttitle = \bfseries,
    breakable = true]
\begin{dfn}
$k=1, \ldots, m=\dim M$となる自然数とする. 
\begin{itemize}
 \setlength{\parskip}{0cm}
  \setlength{\itemsep}{2pt} 
\item 任意の$p \in M$について$\omega_{p} \in \wedge^{k} T_{p}^{*}M$が一つずつ対応しているとき, その対応$\omega = \{ \omega_p\}_{p \in M}$を\underline{$M$上の$k$次微分形式}という.
\item $M$上の$k$次微分形式$\omega$は座標近傍$(U, x_1, \ldots, x_m)$について, ある$U$上の関数$f_{i_1 i_2 \cdots i_k}: U \rightarrow \R$($1 \le i_1< \cdots < i_k \le m$)があって
    $$
    \omega|_{U} = \sum_{ 1 \le i_1< \cdots < i_k\le m }f_{i_1 i_2 \cdots i_k}d x_{i_1}\wedge dx_{i_2} \wedge \cdots \wedge dx_{i_k}
    $$
    とかける. 各座標近傍$(U, x_1, \ldots, x_m)$について上の$f_{i_1 i_2 \cdots i_k}$が$C^{\infty}$級となるとき, $X$は\underline{$C^{\infty}$級$k$次微分形式}であるという.
  
    %$M$上の$C^{\infty}$級ベクトル場の集合を$\mathscr{X}(M)$で表す. 
\end{itemize}
    \end{dfn}
    \end{tcolorbox}
 断りのない限り微分形式は全て$C^{\infty}$級であるとする. 
%以下, 座標近傍$(U, x_1, \ldots, x_m)$とする.

 \begin{tcolorbox}[
    colback = white,
    colframe = green!35!black,
    fonttitle = \bfseries,
    breakable = true]
\begin{dfn}[外積]
%$k=1, \ldots, m=\dim M$となる自然数とする. 

 \setlength{\parskip}{0cm}
  \setlength{\itemsep}{2pt} 
  $M$上の$k$次微分形式$\omega$と$l$次微分形式$\eta$について, その\underline{外積$\omega \wedge \eta $}を
$$
\omega \wedge \eta(X_1, \ldots, X_{k+l})
=
\frac{1}{k! l!} \sum_{\sigma \in S_{k+l}} \sgn(\sigma) \omega(X_{\sigma(1)}, \ldots, X_{\sigma(k)}) \eta(X_{\sigma(k+1)}, \ldots, X_{\sigma(k+l)})
\text{とする.}
$$
$\omega = \sum_{ 1 \le i_1< \cdots < i_k\le m }f_{i_1 \cdots i_k}d x_{i_1}\wedge \cdots \wedge dx_{i_k}$, $\eta = \sum_{ 1 \le j_1< \cdots < j_l\le m }g_{j_1 \cdots j_l}d x_{j_1} \wedge \cdots \wedge dx_{j_l}$と座標近傍$(U, x_1, \ldots, x_m)$上でかけている場合, 
$$
\omega \wedge \eta=
 \sum_{ 1 \le i_1< \cdots < i_k\le m }\sum_{ 1 \le j_1< \cdots < j_l\le m }
 f_{i_1 \cdots i_k}g_{j_1 \cdots j_l}d x_{i_1} \wedge \cdots \wedge dx_{i_k}\wedge d x_{j_1} \wedge \cdots \wedge dx_{j_l}
\text{となる. }
$$
    \end{dfn}
    \end{tcolorbox}
    
  \begin{tcolorbox}[
    colback = white,
    colframe = green!35!black,
    fonttitle = \bfseries,
    breakable = true]
\begin{dfn}[外微分]
%$k=1, \ldots, m=\dim M$となる自然数とする. 
%座標近傍$(U, x_1, \ldots, x_m)$とする
$M$上の$k$次微分形式$\omega$について, \underline{外微分$d \omega$}を
\begin{align*}
\begin{split}
d\omega(X_1, \ldots, X_{k+1})
&=
\sum_{i=1}^{k+1}(-1)^{i+1}X_i(\omega(X_1, \ldots,\widehat{X_{i}}, \ldots, X_{m}))\\
&+\sum_{i<j}(-1)^{i+j}\omega([X_i, X_j], X_1, \ldots, \widehat{X_{i}},  \ldots, \widehat{X_{j}}, \ldots, X_{m}).
%\text{とする.}
\end{split}
\end{align*}
とする.  ここで$(X_1, \ldots, X_{k+1})$はベクトル場とし, $(X_1, \ldots,\widehat{X_{i}}, \ldots, X_{m})$は$(X_1, \ldots,X_{i-1}, X_{i+1}, \ldots, X_{m})$を意味する.
$\omega = \sum_{ 1 \le i_1< \cdots < i_k\le m }f_{i_1 \cdots i_k}d x_{i_1}\wedge \cdots \wedge dx_{i_k}$と座標近傍$(U, x_1, \ldots, x_m)$上でかけている場合, 
\begin{align*}
\begin{split}
d\omega
&=
 \sum_{ 1 \le i_1< \cdots < i_k\le m }
 df_{i_1 \cdots i_k}d x_{i_1} \wedge \cdots \wedge dx_{i_k}\\
& =  \sum_{ 1 \le i_1< \cdots < i_k\le m }\left(\sum_{j=1}^{m}\pdrv{f_{i_1 \cdots i_k}}{x_{j}}d x_{j}\right)\wedge d x_{i_1} \wedge \cdots \wedge dx_{i_k}
\text{となる. }
\end{split}
\end{align*}

    \end{dfn}
    \end{tcolorbox}
      \begin{tcolorbox}[
    colback = white,
    colframe = green!35!black,
    fonttitle = \bfseries,
    breakable = true]
\begin{dfn}[引き戻し]
%$k=1, \ldots, m=\dim M$となる自然数とする. 

 $\varphi : M \rightarrow N$を$C^{\infty}$写像とする. $N$上の$l$次微分形式$\eta$について, $\eta$の$\varphi$による\underline{引き戻し$\varphi^{*}\eta$}を
$$
(\varphi^{*}\eta)_{p}(X_{p}) := \eta_{\varphi(p)}((d\varphi)_{p} X_{p}) \quad (\forall p \in M, \forall X \in T_{p}M)
$$
と定める. これは$M$上の$l$次微分形式となる. 
%これを$\eta$の$\varphi$による引き戻しという.
$M$の座標近傍$(U, x_1, \ldots, x_m)$, $N$の座標近傍$(V, y_1, \ldots, y_n)$に関して, $\eta = \sum_{ 1 \le j_1< \cdots < j_l\le m }g_{j_1 \cdots j_l}d y_{j_1} \wedge \cdots \wedge dy_{j_l}$とかけている場合, 
$$
\varphi^{*}\eta:= 
\sum_{ 1 \le j_1< \cdots < j_l\le m }(g_{j_1 \cdots j_l}\circ \varphi )
\left(\sum_{i_1 =1}^{m}\pdrv{ y_{j_1}}{x_{i_1}} dx_{i_1} \right)\wedge \cdots \wedge 
\left(\sum_{i_l =1}^{m}\pdrv{dy_{j_l}}{x_{i_l}} dx_{i_l}\right)\text{となる.}
$$
    \end{dfn}
    \end{tcolorbox}
 
 %\begin{rem}上の定義は局所座標$(U, x_1, \ldots, x_m)$を用いていない定義である. 局所座標を用いない定義はわかりずらい印象がある.(外微分は特に局所座標の方がわかりやすいと思う). ただ証明などではこちらが便利な時もある. \end{rem}
 
 \begin{exa}
 $\omega=f_1dx_1+ f_2dx_2$,  $\eta=g_1dx_1+ g_2dx_2$, $\varphi (z_1, z_2) = (\varphi_1(z_1,z_2), \varphi_2(z_1,z_2))$とすると外積, 外微分, 引き戻しはそれぞれ次の通りとなる. 
 $$
 \omega \wedge \eta = (f_1dx_1+ f_2dx_2) \wedge (g_1dx_1+ g_2dx_2)
 = (f_1g_2)dx_1 \wedge dx_2 + (f_2g_1)dx_2 \wedge dx_1 = (f_1g_2 - f_2 g_1) dx_1 \wedge dx_2.
 $$
 $$
 d \omega = \left(\pdrv{f_1}{x_1} dx_1+ \pdrv{f_1}{x_2} dx_2\right) \wedge dx_1 + \left(\pdrv{f_2}{x_1} dx_1+ \pdrv{f_2}{x_2} dx_2\right) \wedge dx_2
 %= \pdrv{f_1}{x_2} dx_2\wedge dx_1 +  \pdrv{f_2}{x_1} dx_1\wedge dx_2
= \left( -\pdrv{f_1}{x_2} + \pdrv{f_2}{x_1}\right)dx_1 \wedge dx_2.
 $$
 \begin{align*}
\begin{split}
 \varphi^{*}\omega
& =
 f_{1} (\varphi(z)) d\varphi_{1} +  f_{2} (\varphi(z)) d\varphi_{2}
 =
  f_{1} (\varphi(z)) \left(\pdrv{\varphi_1}{z_1} dz_1 + \pdrv{\varphi_1}{z_2} dz_2 \right) 
  +  f_{2} (\varphi(z)) \left(\pdrv{\varphi_2}{z_1} dz_1 + \pdrv{\varphi_2}{z_2} dz_2 \right). \\
  %&=(f_{1} (\varphi(z)) \pdrv{\varphi_1}{z_1} +  f_{2} (\varphi(z)) d\varphi_{2} (\pdrv{\varphi_2}{z_1})dz_1+ (f_{1} (\varphi(z))\pdrv{\varphi_1}{z_2}+  f_{2} (\varphi(z)) \pdrv{\varphi_2}{z_2} )dz_2
\end{split}
 \end{align*}
 
 \end{exa}


  \begin{tcolorbox}[
    colback = white,
    colframe = green!35!black,
    fonttitle = \bfseries,
    breakable = true]
\begin{prop}
$\omega$を$k$次微分形式, $\eta$を$l$次微分形式, $\zeta$を$s$次微分形式とする. 次が成り立つ.
\begin{itemize}
 \setlength{\parskip}{0cm}
  \setlength{\itemsep}{2pt} 
\item$\omega \wedge \eta = (-1)^{kl} \eta \wedge \omega$, $\omega \wedge (\eta  \wedge \zeta)= (\omega \wedge \eta)  \wedge \zeta$. 
\item $\varphi^{*}(\omega \wedge \eta) = \varphi^{*}(\omega) \wedge \varphi^{*}(\eta)$.
\item $d(\omega \wedge \eta ) = (d \omega) \wedge \eta + (-1)^{k}\omega \wedge (d \eta)$.
    %$M$上の$C^{\infty}$級ベクトル場の集合を$\mathscr{X}(M)$で表す. 
    \item $d(d \omega)=0$, $d(\varphi^{*}\omega)=\varphi^{*}(d \omega)$.
\end{itemize}
    \end{prop}
    \end{tcolorbox}
  
  
    \begin{tcolorbox}[
    colback = white,
    colframe = green!35!black,
    fonttitle = \bfseries,
    breakable = true]
\begin{dfn}[ド・ラームコホモロジー]
$k$次微分形式の集合を$\Omega^{k}(M)$とする. 
$$
0 \rightarrow \Omega^{0}(M) \overset{d}{\rightarrow}\Omega^{1}(M) \overset{d}{\rightarrow} \cdots 
\overset{d}{\rightarrow} \Omega^{m}(M) \overset{d}{\rightarrow} 0
$$
を\underline{ド・ラーム複体}といい, 
$$
H_{DR}^{k}(M):= \frac{\ker (d : \Omega^{k}(M) \rightarrow  \Omega^{k+1}(M))}{\Im(d : \Omega^{k-1}(M) \rightarrow  \Omega^{k}(M)) }
%\{ \omega \in\Omega^{k}(M) | d \omega =0 \}/\{ \eta \in \Omega^{k}(M) | \exists \zeta  \in \Omega^{k}(M) \text{s.t.} \eta = d \zeta \}
$$
を$M$の\underline{$k$次のド・ラームコホモロジー群}という.


    \end{dfn}
    \end{tcolorbox}
    
\begin{rem}
$d\omega=0$なる微分形式を\underline{閉形式}といい, ある微分形式$\eta$があって$\omega = d \eta$とかける微分形式$\omega$を\underline{完全形式}という. $d \circ d =0$なので完全形式ならば閉形式である.  ド・ラームコホモロジー群は閉形式と完全形式のずれを記述している群である.
\end{rem}

    
 \section{1の分割と多様体上の積分}
 
  \begin{tcolorbox}[
    colback = white,
    colframe = green!35!black,
    fonttitle = \bfseries,
    breakable = true]
\begin{thm}
$p \in M$と$p$の開近傍$U$について, ある$p$の開近傍$V$と$C^{\infty}$級関数$\rho : M \rightarrow \R$があって
$\overline{V} \subset U$かつつぎを満たす.
$$
\left\{
\begin{array}{ll}
\rho(q) =1& q \in \overline{V} \\
0 \leqq \rho(q) <1& q \in U \setminus \overline{V}\\
\rho(q)=0&q \in M \setminus U
\end{array}
\right.
$$
特に$\rho$の台${\rm Supp}(\rho) :=\overline{\{q \in M | \rho(q) \neq 0 \}}$とするとき, $\Supp(\rho) \subset U$となる.
     \end{thm}
    \end{tcolorbox} 
    
 \begin{tcolorbox}[
    colback = white,
    colframe = green!35!black,
    fonttitle = \bfseries,
    breakable = true]
\begin{thm}[1の分割]
$M$が第二可算であると仮定する.
任意の$M$の開被覆$\{U_{\alpha}\}_{\alpha \in A}$についてある可算個の$C^{\infty}$級関数$\rho_{j} : M \rightarrow \R$($j \in \N$)があって次が成り立つ
\begin{enumerate}
 \setlength{\parskip}{0cm}
  \setlength{\itemsep}{2pt} 
\item $\{ \Supp(\rho_{j})\}_{j \in \N}$は$M$の被覆であり, $p \in M$についてある$p$の開近傍$U$をとれば$U \cap \Supp(\rho_{j}) \neq \varnothing$なる$j$は有限個になる.(局所有限な被覆という.)
\item 任意の$j \in \N$についてある$\alpha_{j} \in A$があって$\Supp(\rho_{j}) \subset U_{\alpha_j}$となる. ($\{U_{\alpha}\}_{\alpha \in A}$の細分という.)
\item $0 \le \rho_j \le 1$かつ$\sum_{j \in \N} \rho_{j} \equiv 1$.
\end{enumerate}
この$\{ \rho_{j} \}_{j \in \N}$を\underline{$\{U_{\alpha}\}_{\alpha \in A}$に従属する1の分割}という.
     \end{thm}
    \end{tcolorbox} 
\begin{rem}
上は$\sigma$コンパクトで成り立つ定理である.(第二可算な多様体は$\sigma$コンパクトであるらしい.)ただ$\sigma$コンパクトは応用上で使うか怪しいし, 多様体に第二可算を仮定することが多いので, ここでは第二可算として主張を述べた.\footnote{「トゥー 多様体」では多様体に第二可算を仮定している.}要するに1の分割は取れると思って良い. 
\end{rem}

 %この章では多様体$M$について第二可算を仮定する. 第二可算ならばsigma compact
 
  \begin{tcolorbox}[
    colback = white,
    colframe = green!35!black,
    fonttitle = \bfseries,
    breakable = true]
\begin{dfn}
\label{integral_local}
$(U, \varphi) = (U, x_1, \ldots, x_m)$を座標近傍とし, $U$上の$m$次微分形式を$\omega = f(x_1, \ldots, x_m)dx_1 \wedge \cdots \wedge dx_m$とする.
$\varphi(U)$が正方形領域$V:=[-a,a]^{m}$に含まれるとき, $\omega$の$U$上の積分を
$$
\int_{U} \omega := \int_{[-a,a]^{m}}f(x_1, \ldots, x_m)dx_1 \wedge \cdots \wedge dx_m
\text{で定義する.}
$$
    \end{dfn}
    \end{tcolorbox} 
    
 \begin{tcolorbox}[
    colback = white,
    colframe = green!35!black,
    fonttitle = \bfseries,
    breakable = true]
\begin{dfn}

\begin{itemize}
 \setlength{\parskip}{0cm}
  \setlength{\itemsep}{2pt} 
\item $(U, x_1, \ldots, x_m)$, $(V, y_1, \ldots, y_m)$を$U \cap V \neq \phi$となる$M$の座標近傍とする.
$(U, x_1, \ldots, x_m)$と$(V, y_1, \ldots, y_m)$が\underline{同じ向き}であるとは, $U \cap V$上で
$$
\pdrv{(y_1, \ldots, y_m)}{(x_1, \ldots, x_m)}:=\det(\left( \pdrv{y_j}{x_i} \right)_{1\le i,j \le m}) >0
\text{となることとする.}
$$
\item \underline{$M$が向きづけ可能}であるとは, $M$の座標近傍系$\{ (U_{\alpha}, x_{1}^{\alpha}, \ldots, x_{m}^{\alpha})\}$であって, 同じ向きになるものが存在することとする.
\end{itemize}
    \end{dfn}
    \end{tcolorbox} 

  \begin{tcolorbox}[
    colback = white,
    colframe = green!35!black,
    fonttitle = \bfseries,
    breakable = true]
\begin{thm}
$M$が向きづけ可能なコンパクト$m$次元多様体とし, $\omega$を$m$次微分形式とする.
このとき同じ向きになる$M$の座標近傍系$U_1, \ldots, U_N$とそれに従属する1の分割$\rho_{1}, \ldots, \rho_{N}$があって,
$\omega$の$M$上の積分を
$$
\int_{M} \omega := \sum_{j=1}^{N} \int_{M} \rho_j \omega
$$
で定義する. この積分の値は実数値であり, 1の分割や近傍系の取り方によらない. 
    \end{thm}
    \end{tcolorbox} 

\begin{rem}
 $\rho_j \omega$は定義\ref{integral_local}の仮定を満たすため上のように積分が定義できる. 
$M$がコンパクトでない場合でも1の分割が取れれば積分は定義できるが, 有限の値になるかはわからない. 
\end{rem}

 \section{ストークスの定理}
 以下の内容は「坪井俊 著 幾何学3 微分形式」を参考にした.

    \begin{tcolorbox}[
    colback = white,
    colframe = green!35!black,
    fonttitle = \bfseries,
    breakable = true]
    \begin{dfn}[]
    $M$を第二可算ハウスドルフ空間とする. 次の条件が成り立つとき$M$は$m$次元境界つき($C^{\infty}$級)多様体と呼ばれる.
     \begin{enumerate}
     \setlength{\parskip}{0cm}
  \setlength{\itemsep}{2pt} 
     \item $M$の開被覆$M = \cup_{\lambda \in \Lambda} U_{\lambda}$と像への同相写像
     $$
     \varphi_{\lambda} : U_{\lambda} \rightarrow \mathbb{H}^n := \{ (x_1, x_2, \ldots, x_m) \in \R^m | x_1 \geqq 0\}
     \text{が存在する.}
     $$
    % 座標近傍系$\{(U_\lambda, \varphi_\lambda)\}_{\lambda \in\Lambda}$があって, $M = \cup_{\lambda \in \Lambda} U_{\lambda}$となる. 
     \item $U_\lambda \cap U_\mu \neq \phi$なる$\lambda, \mu \in \Lambda$について
    $
   \varphi_\mu\circ \varphi_{\lambda}^{-1} : \varphi_{\lambda}(U_\lambda \cap U_\mu) \rightarrow \varphi_{\mu}(U_\lambda \cap U_\mu) 
    $
    は$C^{\infty}$級写像である
     \end{enumerate}
 $\partial M := \cup_{\lambda \in \Lambda} \varphi_{\lambda}^{-1}(\{ 0\} \times \R^{m-1}) \subset $を$M$の境界と呼ぶ.
    \end{dfn}
    \end{tcolorbox}   
\begin{rem}
$M$の境界 $\partial M$は$m-1$次元多様体となる. また$M$が向きづけ可能であるとき, $\partial M$には座標近傍系$\{(U_\lambda, x_{2}^{\lambda}, \ldots, x_{m}^{\lambda})\}_{\lambda \in \Lambda}$によって向きが入る.
\end{rem}

  \begin{tcolorbox}[
    colback = white,
    colframe = green!35!black,
    fonttitle = \bfseries,
    breakable = true]
\begin{thm}
$M$が向きづけ可能なコンパクト$m$次元境界つき多様体とし, $\eta$を$m-1$次微分形式とするとき, 次が成り立つ. 
$$
\int_{M} d \eta = \int_{\partial M} \eta 
$$
    \end{thm}
    \end{tcolorbox} 
    
ストークスの定理は境界がない多様体(つまり普通の意味での"多様体")について述べると次のとおりである.
「$M$が向きづけ可能なコンパクト$m$次元多様体とし, $\eta$を$m-1$次微分形式とするとき, $\int_{M} d \eta =0 $となる.」
ストークスの定理は研究でも応用でも使われる定理である. 
\newpage



\section{演習問題}
問題の上に$^{\bullet}$がついている問題は\underline{解けてほしい}問題である. 問題の上に$^{*}$がついている問題は\underline{面白いかちょっと難しい}問題である. 

以下断りがなければ$M,N$は$C^{\infty}$級多様体とし, $m = \dim M$とする.
$\R^n$をユークリッド空間とし, $S^n \subset \R^{n+1}$を半径1の$n$次元球面とする.

\vspace{11pt}
\hspace{-22pt}{\large $\bullet$ 微分形式の問題}
\begin{enumerate}[label=\textbf{問}3.\arabic*]

\item $^{\bullet}$ 講義で配られるプリントにある微分形式に関する計算問題を好きなだけ解答せよ. \footnote{この問題は解答者が複数いても良い. なるべく糟谷先生のプリントの問題も解いてください. }

\item $^{\bullet}$ $\R^{2n}$上の2次微分形式$\omega = \sum_{i=1}^{n} dx_{2i-1} \wedge dx_{2i}$について$\omega^n$を求めよ.

\item $^{\bullet}$ $i : S^2 \rightarrow \R^3$を包含写像とする. 次の問いに答えよ.
\begin{enumerate}
 \setlength{\parskip}{0cm}
  \setlength{\itemsep}{2pt} 
\item $i^{*}(dx \wedge dy \wedge dz)$を求めよ.
\item $i^{*}(dx \wedge dy)$の値が0になる$S^2$の点を全て求めよ.
\end{enumerate}


\item (多様体の基礎 19章) $f : M \rightarrow \R$を$C^{\infty}$級関数とする. 
\begin{enumerate}
 \setlength{\parskip}{0cm}
  \setlength{\itemsep}{2pt} 
\item 微分写像$df_{p} : T_{p}M \rightarrow T_{f(p)}\R \cong \R$により, $df:= \{df_{p}\}_{p \in M}$は$M$上の微分形式だと思える.
 座標近傍$(U, x_1, \ldots, x_m)$を用いて, 微分形式$df$は次のように表せることを示せ. 
 $$
 df|_{U} = \pdrv{f}{x_1} dx_1 + \cdots + \pdrv{f}{x_m}dx_{m}
 $$
\item $X$をベクトル場とするとき, $(df) (X) = X(f)$を示せ.  %\footnote{定義から求める方法と局所座標を用いて示す方法の2種類がある.}
%$p \in M$について$(df)_p (X_{p}) = X_{p}(f)$が成り立つことを示せ. \footnote{定義から求める方法と局所座標を用いて示す方法の2種類がある.}
\end{enumerate}




\item $V$を$\R$上の$m$次元ベクトル空間とし, $\{ e_1, \ldots, e_m\}$を$V$の基底とし, $\{ \omega_1, \ldots, \omega_m\}$は$V^{*}$を $\{ e_1, \ldots, e_m\}$の双対基底とする. また$k$を2以上の自然数とする. 次の問いに答えよ
\begin{enumerate}
 \setlength{\parskip}{0cm}
  \setlength{\itemsep}{2pt} 
\item %$\{ \omega_1, \ldots, \omega_m\}$を用いて
$k$次多重線型形式のなす空間$\otimes^{k} V^{*}$の基底を一つ構成せよ. また$\otimes^{k} V^{*}$の次元を求めよ.
\item $k$次対称形式のなす空間${\rm S}^{k}(V^{*})$の基底を一つ構成せよ. また${\rm S}^{k}(V^{*})$の次元を求めよ.
\item $k$次交代形式のなす空間$\wedge^{k} V^{*}$の基底を一つ構成せよ. また$\wedge^{k} V^{*}$の次元を求めよ.
\end{enumerate}

%\item (多様体の基礎 20章) $\omega$を1次微分形式, $X,Y$を$M$上のベクトル場とするとき$$d \omega(X,Y) = X(\omega(Y)) - Y(\omega(X)) -\omega([X,Y])\text{を示せ.}$$


\item $k$を1以上の自然数とする. $\wedge^{k}T^{*}M = \cup_{p \in M}\wedge^{k}T_{p}^{*}M$に$C^{\infty}$級多様体の構造が入ることを示せ. またその多様体の次元を求めよ. \footnote{難しければ$k=1,m$の場合のみ解答しても良い. $T^{*}M$は\underline{余接ベクトル束}と呼ばれる. }

\item $^{*}$ $X = \R^3 \setminus \{(0,0,0)\}$とし, $f(x,y,z)$を$X$上の$C^{\infty}$級関数で$r = \sqrt{x^2 + y^2 + z^2}$を用いて
$f(x,y,z) = h(r)$とかけているとする.
$X$上の1次微分形式$\omega$を
$$
\omega = f(x,y,z)(x dx + y dy + z dz)
$$
とする. 次の問いに答えよ.
\begin{enumerate}
 \setlength{\parskip}{0cm}
  \setlength{\itemsep}{2pt} 
\item $d\omega =0$を示せ. つまり$\omega$は閉形式である.
\item ある$C^{\infty}$級関数$g$があって$\omega =dg$となることを示せ.つまり$\omega$は完全形式である.
\item $\Delta \varphi=0$となる$C^{\infty}$級関数$\varphi$によって$\omega =d\varphi$となるとき, $f$を$x,y,z$を用いて表せ. 
ここで
$$
\Delta \varphi=\pdrv{^2\varphi}{x^{2}}+\pdrv{^2\varphi}{y^{2}}+\pdrv{^2\varphi}{z^{2}}
\text{である.}
$$
\end{enumerate}

\item $^{\bullet}$ $\R^{2} \setminus \{0\}$上の1次微分形式
$$
\omega = \frac{-ydx + x dy}{x^2+y^2}
$$
について次の問いに答えよ.
\begin{enumerate}
 \setlength{\parskip}{0cm}
  \setlength{\itemsep}{2pt} 
\item 極座標$(x,y)=(r \cos \theta, r \sin \theta)$を用いて$\omega$を$dr,d\theta$で表せ. 
\item $d \omega=0$を示せ. つまり$\omega$は閉形式である. 
\item $C=S^{1}$とし, 向きを反時計回りで入れる. $\int_{C} \omega $を計算せよ. 
\item $\omega = d g$となる$C^{\infty}$級関数は存在しないことを示せ. つまり$\omega$は完全形式ではない. 
\end{enumerate}

\item 連結な多様体$M$について0次ド・ラームコホモロジー群$H^{0}_{DR}(M)$を求めよ. 

\item \label{poincare}  $^{*}$ 次の問いに答えよ.
\begin{enumerate}
\setlength{\parskip}{0cm}
  \setlength{\itemsep}{2pt} 
\item $\omega$を$d \omega=0$となる$\R^2$の$1$次微分形式とする.
$(x,y) \in \R^2$について$L_{(x,y)}$を0が始点で$(x,y) $が終点となる線分とし, 
$
g(x,y) = \int_{L_{(x,y) }} \omega %\quad (x \in \R^n)
$
とおく. このとき$g(x,y)$は$\omega=dg$となる$\R^2$上の$C^{\infty}$級関数であることを示せ. 
\item $\R^2$の$2$次微分形式$\eta$についてある$1$次微分形式$\omega$があって$\eta = d \omega$となることを示せ. 
\item $\R^2$のド・ラームコホモロジー群$H^{k}_{DR}(\R^2)$($k=0,1,2,\ldots$)を求めよ. \footnote{余裕があれば$H^{k}_{DR}(\R^n)$はどうなるか考察せよ. }
\end{enumerate}

\item  $^{*}$ $S^1$のド・ラームコホモロジー群$H^{k}_{DR}(S^1)$($k=0,1,2,\ldots$)を求めよ.\footnote{頑張れば今の状況でも求められる.(トゥー 多様体24章を見よ.) 他にも\underline{ド・ラームの定理}「滑らかな多様体$M$について$H^{k}_{DR}(M) \cong {\rm Hom}(H_{k}(M,\Z), \R)$が成り立つ」を認めればホモロジー群から求められる. }

%ここで$L_x$は0が始点で$x$が終点とする線分とする. 


%\item\label{poincare}  $^{*}$ $\R^n$のド・ラームコホモロジー群$H^{k}_{DR}(\R^n)$($k=0,1,2,\ldots$)を求めよ.
\item 
 $\R^3$の関数(スカラー場)$F(x,y,z)$とベクトル場${\bf V}(x,y,z) = (V_1(x,y,z), V_2(x,y,z), V_3(x,y,z))$について, 
$$
{\rm grad}(F)=\nabla F=\left(\pdrv{F}{x}, \pdrv{F}{y}, \pdrv{F}{z}\right) \quad {\rm div}({\bf V}) = \nabla \cdot {\bf V} = \pdrv{V_1}{x}+\pdrv{V_2}{y}+ \pdrv{V_3}{z}
$$
$$
{\rm rot}({\bf V})=\nabla \times {\bf V}
=\left(\pdrv{V_3}{y}- \pdrv{V_2}{z}, \pdrv{V_1}{z} - \pdrv{V_3}{x}, \pdrv{V_2}{x} - \pdrv{V_1}{y}\right)
$$
%$${\rm div}({\bf V}) = \nabla \cdot {\bf V} = \pdrv{V_1}{x}+\pdrv{V_2}{y}+ \pdrv{V_3}{z}$$
と定義する. 次の問いに答えよ.\footnote{この問題は「$\R^3$上のベクトル解析が微分形式によって再解釈される」ことを確かめる問題である.そのため数学的な記述は少々曖昧であるのでご了承いただきたい.}
\begin{enumerate}
 \setlength{\parskip}{0cm}
  \setlength{\itemsep}{2pt} 
\item 下の図式が可換になるように$\Phi_1,\Phi_2, \Phi_3$をうまく定義せよ. %次の図式を考える.
\begin{equation*}
\hspace{-55pt}
\xymatrix@C=30pt@R=20pt{
\{\text{関数(スカラー場)}\}\ar@{=}[d]\ar@{->}[r]^{\hspace{15pt}{\rm grad}} &\{\text{ベクトル場}\}\ar@{->}[r]^{{\rm rot}}  \ar@{->}[d]^{\Phi_1}
&\{\text{ベクトル場} \}\ar@{->}[d]^{\Phi_2}\ar@{->}[r]^{{\rm div}\hspace{15pt}} &\{\text{関数(スカラー場)}\}  \ar@{->}[d]^{\Phi_3}\\ 
\{\text{関数(0次微分形式)}\}\ar@{->}[r]^{\hspace{15pt} d}&\{\text{1次微分形式}\} \ar@{->}[r]^{d}
&\{ \text{2次微分形式}\}\ar@{->}[r]^{d}&\{ \text{3次微分形式}\}\\
 }
\end{equation*}
%上の図式が可換になるようにうまく$\Phi_1,\Phi_2, \Phi_3$を定義せよ.
\item ${\rm rot}({\rm grad}(F)) =0$と${\rm div}({\rm rot}({\bf V}) =0$をそれぞれ示せ.
\item $\R^3$のベクトル場${\bf V}$について, ${\rm rot}{\bf V}\equiv 0$であることは${\bf V} = {\rm grad}\phi$なるスカラー場(スカラー・ポテンシャル)$\phi$が存在することと同値であることを示せ.\footnote{ヒント: $k >0$について$H^{k}_{DR}(\R^3)=0$であることを用いる. 同様に${\rm div}{\bf V}\equiv 0$であることは${\bf V} = {\rm rot}{\bf A}$なるベクトル場(ベクトル・ポテンシャル)${\bf A}$が存在することと同値であることがわかる. }
%\item ${\rm div}{\bf V}\equiv 0$であることは${\bf V} = {\rm rot}{\bf A}$なるベクトル場(ベクトル・ポテンシャル)${\bf A}$が存在することと同値であることを示せ. 
\end{enumerate}

\newpage 
\item $^{}$ [Tu. Problem 19.13] 次を英訳し問題に解答せよ.
%[Tu. Problem 19.13 Twentieth-century formulation of Maxwell’s equations] 次を英訳し問題に解答せよ.

In Maxwell's theory of electricity and magnetism, developed in the late nineteenth century,
the electric field ${\bf E} = (E_1, E_2, E_3)$ and the magnetic field ${\bf B} = (B_1, B_2, B_3)$ in a vacuum
$\R^3$
with no charge or current satisfy the following equations:
$$
{\rm rot}{\bf E} = - \pdrv{{\bf B}}{t} ,  
{\rm rot} {\bf B} = \pdrv{{\bf E}}{t}, 
{\rm div} {\bf E} = 0,
{\rm div} {\bf B} = 0. 
$$
We define the 1-form $E$ on $\R^3$ corresponding to the vector field $ { \bf E}$ by $E = E_1 dx + E_2 dy + E_3 dz$ and define the 2-form $B$ on $\R^3$ corresponding to the vector field ${ \bf B}$ by $B = B_1 dy \wedge dz + B_2 dz \wedge dx + B_3 dx \wedge dy$. 
%By the correspondence in Subsection 4.6, the 1-form $E$ on $\R^3$ corresponding to the vector field $ { \bf E}$ is $E = E_1 dx + E_2 dy + E_3 dz$ and the 2-form $B$ on $\R^3$ corresponding to the vector field ${ \bf B}$ is $B = B_1 dy \wedge dz + B_2 dz \wedge dx + B_3 dx \wedge dy$. 

Let $\R^4$ be space-time with coordinates $(x, y, z, t)$. 
Then both $E$ and $B$ can be viewed as
differential forms on $\R^4$. Define $F$ to be the 2-form
$$
F = E \wedge dt + B
$$
on space-time. Decide which two of Maxwell's equations are equivalent to the equation $dF =0$.
Prove your answer. \footnote{この文章には続きがあった. "The other two are equivalent to $d * F = 0$ for a star-operator $*$ defined indifferential geometry."つまり後二つは$d * F =0$と同じである. ここで$*$はHodge-star operatorである. }


\vspace{11pt}
\hspace{-33pt}{\large $\bullet$1の分割・多様体上の積分・ストークスの定理}

\item %$^{}$[Tu Problem 13.3 Smooth Urysohn Lemma] 
Let $A$ and $B$ two disjoint closed sets in a manifold $M$. Find $C^{\infty}$ function $f$ on $M$ such that $f$ is identically 1 on $A$ and identically 0 on $B$.
%\begin{enumerate}
%\setlength{\parskip}{0cm}
  %\setlength{\itemsep}{2pt} 
 %\item Let $A$ and $B$ two disjoint closed sets in a manifold $M$. Find $C^{\infty}$ function $f$ on $M$ such that $f$ is identically 1 on $A$ and identically 0 on $B$.
%\item Let $A$ be a closed subset and $U$ an open subset of a manifold $M$. Show that there is a $C^{\infty}$function $f$ on $M$ such that $f$ is identically 1 on $A$ and $\Supp f \subset U$.
%\end{enumerate}

\item $M$を向きづけ可能なコンパクト$m$次元多様体とし, $N $を$m-1$次元の$M$の閉部分多様体とする. $\omega$を$m$次微分形式とするとき
$
\int_{M} \omega = \int_{M \setminus N} \omega
\text{を示せ.}
$
%であることを示せ.
\item $^{\bullet}$(多様体の基礎 20章) リーマン球面$\C\mathbb{P}^1 = \C \cup \C$を構成する2つの複素平面$\C$をそれぞれ$z = x + iy$, $\xi = \zeta + i\eta$の対応で$(\zeta, \eta)$平面, $(x,y)$平面と同一視する. 次の問いに答えよ.
\begin{enumerate}
 \setlength{\parskip}{0cm}
  \setlength{\itemsep}{2pt} 
\item 座標変換$z= \frac{1}{\xi}$を$(\zeta, \eta)$と$(x,y)$を用いて表せ.
\item $(x,y)$平面上の2次微分形式$\omega = \frac{dx \wedge dy}{(1+x^2+y^2)^2}$は$\C\mathbb{P}^1$上の2次微分形式$\widetilde{\omega}$に拡張できることを示せ.
%$(\zeta, \eta)$平面上の$C^{\infty}$級2次微分形式$\widetilde{\omega}$に拡張できることを示せ
\footnote{ヒント: $(\zeta, \eta)$平面上の$C^{\infty}$級2次微分形式$\alpha$で, $(\zeta, \eta)$平面と$(x,y)$平面の共通部分で$\omega$と一致するものを一つ見つけよ. そうすると$\alpha$と$\omega$の貼り合わせで$\widetilde{\omega}$が構成できる.}
\item $\int_{\C\mathbb{P}^1 } \widetilde{\omega}$の値を求めよ.
\end{enumerate}

\item 2次元トーラス$T^2 = \{(x,y,z,w) \in \R^4 | x^2 + y^2 = z^2 + w^2 =1\}$について
$$
\int_{T^2} yzw \, dx \wedge dz
$$
を求めよ. ただし$T^2$にどのような向きを入れたのかを明記すること.

\item $^{\bullet}$ $$\int_{S^2} x dy \wedge dz + y dz \wedge dx + z dx \wedge dy$$を求めよ. ただし$S^2$にどのような向きを入れたのかを明記すること.
\item  $^{\bullet}$ $D=[a,b] \times [c, d]$とし, $f(x,y), g(x,y)$を$D$上の$C^{\infty}$級関数とする.\footnote{$D^{\circ}$上で$C^{\infty}$級で$D$上で連続と言った方がいい?}グリーンの定理
$$
\int_{\partial D} f(x,y) dx + g(x,y) dy = \iint_{D} \left(\pdrv{g}{x}  -\pdrv{f}{y} \right) dxdy
$$
を示せ. ただし$\partial D$にどのような向きを入れたか明記すること.

\item $^{\bullet}$ $\R^{2} \setminus \{ (0,0)\}$上で定義された領域上で定義された関数$f(x,y) = \log\sqrt{x^2 +y^2}$を考える. $\R^{2}  \setminus  \{ (0,0)\}$上の1次微分形式を
$
\omega := \pdrv{f}{x}dy -  \pdrv{f}{y}dx
$
とする. 次の問いに答えよ.
\begin{enumerate}
 \setlength{\parskip}{0cm}
  \setlength{\itemsep}{2pt} 
\item $\R^{2} \setminus \{ (0,0)\}$上 $\Delta f = \pdrv{^2f}{x^2} + \pdrv{^2f}{y^2} =0$であることを示せ.
\item $C_1$を中心$(3,0)$で半径2の円周とし, 向きを反時計回りに入れる. $\int_{C_1} \omega$を計算せよ.
\item $C_2$を中心$(1,0)$で半径4の円周とし, 向きを反時計回りに入れる. $\int_{C_2} \omega$を計算せよ.
\end{enumerate}


\vspace{11pt}
\hspace{-33pt}{\large $\bullet$向きづけの問題}

\item \label{nform} $^{*}$(多様体の基礎 20章)
 多様体$M$が向きづけ可能であるための必要十分条件は, どの点でも0にならない$m$次微分形式$\omega$が存在することであることを示せ.(つまり向きづけ可能とは$\wedge^{m}T^{*}M$が自明になることと同値である.)
%多様体$M$が向きづけ可能であるための必要十分条件は, $\wedge^{m} T^{*}M$
%\end{enumerate}

%\item $\C\mathbb{P}^2$は向きづけ可能であることを示せ.\footnote{より一般に複素多様体は向きづけ可能であることがわかる.}


\item $^{*}$多様体$M$についてその接ベクトル束$TM$は常に向きづけ可能であることを示せ.

\item $^{*}$ $\C\mathbb{P}^n$は向きづけ可能であることを示せ. %\footnote{実はより一般に複素多様体は向きづけ可能である.}
%\item $S^n$は向きづけ可能であることを示せ.

\item $^{*}$ %メビウスの帯を定義し, 向きづけ不可能であることをしめせ. 
$(-1,1) \times \R$に同値関係$\sim$を
	$$
	(x,y) \sim (z,w)\Leftrightarrow \text{ある整数$m$があって}z = (-1)^m x, w = y+m.
	$$
	と定義する. $X := ((-1,1) \times \R) / \sim$とし\underline{メビウスの帯}という. 商写像$\pi : (-1,1) \times \R \rightarrow X$によって$X$に位相を入れる. 次の問いに答えよ.
	\begin{enumerate}
	 \setlength{\parskip}{0cm}
  \setlength{\itemsep}{2pt} 
  \item $U_1: = \pi( (-1,1) \times (0,1) ) $,  $U_2: = \pi( (-1,1) \times (-\frac{1}{2},\frac{1}{2}) )$とおく. 
各$i=1,2$について$\R^2$の開集合$V_i$への同相写像$\varphi_i : U_i \rightarrow V_i$で, $\{ (U_1, \varphi_1), (U_2, \varphi_2)\}$が$X$の座標近傍系になるような$\varphi_1, \varphi_2$を一つ構成せよ. またメビウスの帯は$C^{\infty}$級多様体になることを示せ.
  \item メビウスの帯$X$は向きづけ不可能であることを示せ.
	\end{enumerate}



\item $^{**}$ $\R \mathbb{P}^n$は$n$が奇数なら向きづけ可能であるが, $n$が偶数なら向きづけ不可能であることを示せ. \footnote{一応\ref{nform}を使えば現時点でも求められる. 他にも「$m$次元コンパクト連結多様体$M$について, その$n$次ホモロジー群$H_{n}(X,\Z)$が$\Z$ならば向きづけ可能であり, $0$ならば向きづけ不可能である」という定理を使えば, ホモロジー群からも求められる. }




%\vspace{11pt}\hspace{-33pt}{\large $\bullet$教育的な問題}

\vspace{11pt}
\hspace{-33pt}{\large $\bullet$発展課題\footnote{教育的な問題からそうでない問題まで揃えており, 余裕のある人向けの問題となっております.}}
%\hspace{-22pt}以下の問題は私が少々気になった事柄である. 余裕のある人向けの問題となっております. \footnote{教育的な問題からそうでない問題まで揃えております.}



%\item 複素多様体は剥きづけ可能
\item $^{*}$ $X$をベクトル場とし, $\omega$を$k$次微分形式とする.
$$
(L_{X}\omega) (X_1,\ldots, X_k):=X(\omega(X_1, \ldots, X_k)) - \sum_{i=1}^{k}\omega(X_1, \ldots, [X,X_i], \ldots, X_k)
$$
と定義する. \underline{$L_{X}\omega$を$\omega$の$X$によるLie微分}という. 次の問いに答えよ. \footnote{難しければ$k=1$など低い次数の微分形式対して示して良い. }
\begin{enumerate}
 \setlength{\parskip}{0cm}
  \setlength{\itemsep}{2pt} 
\item $L_X \omega$は$k$次微分形式であることを示せ. 
\item $\{ \varphi_{t} \}_{t \in \R}$を$X$の1パラメーター変換群とするとき, $L_{X} \omega = \lim_{t \rightarrow 0}\frac{\varphi^{*}_{t} \omega - \omega}{t}$を示せ. 
\item $L_{X}L_{Y} - L_{Y}L_{X}= L_{[X,Y]}$を示せ.
\item $L_{X}(\omega \wedge \eta )=L_{X}(\omega) \wedge \eta  + \omega \wedge L_{X}( \eta )$と$d L_{X} = L_{X}  d$をそれぞれ示せ.
\end{enumerate}

\item $^{*}$ $X$をベクトル場とし, $\omega$を$k$次微分形式とする.
$$
(i_{X}\omega) (X_1,\ldots, X_{k-1}):=\omega(X, X_1, \ldots, X_{k-1}) 
$$
と定義する. \underline{$i_{X}\omega$を$\omega$と$X$の内部積}という. 次の問いに答えよ
\begin{enumerate}
 \setlength{\parskip}{0cm}
  \setlength{\itemsep}{2pt} 
\item $i_X \omega$は$k-1$次微分形式であることを示せ. 
%\item $\{ \varphi_{t} \}_{t \in \R}$を$X$の1パラメーター変換群とするとき, $L_{X} \omega = \frac{\varphi^{*}_{t} \omega - \omega}{t}$であることを示せ. 
%\item $L_{X}L_{Y} - L_{Y}L_{X}= L_{[X,Y]}$を示せ.
\item $\omega$を$k$次微分形式とするとき, $i_{X}(\omega \wedge \eta )=i_{X}(\omega) \wedge \eta  +(-1)^k \omega \wedge i_{X}( \eta )$を示せ. 
\item $i_{[X,Y]} = L_{X} i_{Y} - i_{Y} L_X$を示せ.
\item Cartanの公式 $L_X = i_{X}  d + d  i_{X} $を示せ. 
\end{enumerate}

\item $^{*}$ $\omega$を$\R^n$上の1次微分形式とし, $S_{\omega}$を$\R^n$のベクトル場$X$で$i_{X}\omega =0$となるものの集合とする.  $d \omega \wedge \omega =0$ならば任意の$X,Y \in S_{\omega}$について$[X,Y] \in S_{\omega}$であることを示せ.
%次が同値であることを示せ.%\footnote{微分形式を用いた葉層の特徴づけである. }
%\begin{enumerate}
% \setlength{\parskip}{0cm}
 % \setlength{\itemsep}{2pt} 
%\item 任意の$X,Y \in S_{\omega}$について$[X,Y] \in S_{\omega}$である.
%\item $d \omega \wedge \omega =0$.
%\end{enumerate}

\item $^{*}$ $M$を向きづけ可能なコンパクト多様体とし$g$を$M$上のリーマン計量とする. 

\begin{enumerate}
 \setlength{\parskip}{0cm}
  \setlength{\itemsep}{2pt} 
\item $\{ (U_{\lambda}, x_{1}^{\lambda}, \ldots, x_{m}^{\lambda})\}_{\lambda \in \Lambda}$を互いに同じ向きになる座標近傍系とする. 
$g_{ij}^{\lambda} =g \left(\pdrv{}{x_{i}^{\lambda}}, \pdrv{}{x_{j}^{\lambda}}\right)$とし, 
$$
\omega_{\lambda} := \sqrt{\left|\det (g_{ij}^{\lambda}) \right|} dx_{1}^{\lambda} \wedge \cdots \wedge d x_{m}^{\lambda}
$$
とおく. $M$上の$m$次微分形式$\omega_{g}$で$\omega_{g}|_{U_{\lambda}} = \omega_{\lambda} $となるものが存在することを示せ. この$\omega_g$は\underline{リーマン計量の体積要素}と呼ばれる.

\hspace{-33pt}以下, $\R^n$に標準的なリーマン計量$g$, つまり$g_{ij} = \delta_{ij}$となる計量$g$を入れる. \footnote{$\delta_{ij}$はクロネッカーのデルタである.}

%\item 半径1の開円盤を$B^2$とし$B^2$に$\R^2$から誘導されるリーマン計量$g_{B^2}$を入れる時, $\int_{B^2} \omega_{g_{B^2}}$を求めよ.
\item $S^1$に$\R^2$から誘導されるリーマン計量$g_{S^1}$を入れる. $\int_{S^1} \omega_{g_{S^1}}$を求めよ.

\item $S^2$に$\R^3$から誘導されるリーマン計量$g_{S^2}$を入れる. $\int_{S^2} \omega_{g_{S^2}}$を求めよ.

%\item 半径1の開球を$B^{3}$とし, $B^{3}$に$\R^3$から誘導されるリーマン計量$g_{B^3}$を入れる. $\int_{B^3} \omega_{g_{B^3}}$を求めよ.

\end{enumerate}

\item $^{**}$ (学部一年の積分の授業で習ったと思われる)曲線の長さや曲面の表面積を求める公式を上のリーマン計量の体積要素を用いて導出せよ. \footnote{学部1年で線分の長さや表面積の公式習ったと思うが, その公式の証明はされていなかったと思う. 実は表面積や線分の長さの公式はリーマン計量の体積要素からわかるものであり, とどのつまり学部3年にしてようやく表面積や曲線の長さが定義できたのである. }
%$$S_{\omega} := \{ X \in \chi(\R^n)  | i_{X} \}$$
%\item ベクトル解析
%\item カルタンの公式
%\item 葉層の問題
%\item 電磁気の法則

\item $^{**}$ベクトル解析におけるガウスの発散定理を調べ, それが(多様体の)ストークスの定理から導かれることを示せ.\footnote{面積分をうまく定義する必要がある. 本当はもっと演習問題ぽく出したかったがどうもリーマン計量が出てくるためうまく問題が作れなかった...} 

\end{enumerate}

\vspace{11pt}
\begin{wrapfigure}{r}[0pt]{0.2\textwidth}
  \centering
 \includegraphics[height=25mm, width=25mm]{stokes.png}
\end{wrapfigure}

演習の問題は授業ページ(\url{https://masataka123.github.io/2022_winter_generaltopology/})にもあります. 
右のQRコードからを読み込んでも構いません.




\begin{comment}

\begin{wrapfigure}{r}[0pt]{0.5\textwidth}
%\begin{flushright}
 \includegraphics[height=17mm, width=17mm]{stokes.png}
 %\caption*{}
%\end{flushright}
\end{wrapfigure}
\section{微分形式の計算規則}
座標近傍$(U, x_1, \ldots, x_m)$とする.\footnote{あるいは$M$を$\R^m$の開集合と考えても良い.}

    \begin{tcolorbox}[
    colback = white,
    colframe = green!35!black,
    fonttitle = \bfseries,
    breakable = true]
\begin{dfn}
$V$を$m$次元の$\R$ベクトル空間とする.
\begin{itemize}
\item $V$上の$k$次多重線型形式とは$\omega : V \times \cdots \times V \rightarrow \R$となる写像で$\omega(v_1, \ldots, v_k)$が各々$X_i$について線型であることとする. $V$上の$k$次多重線型形式なす$m^k$次元のベクトル空間を$\otimes^{k} V^{*}$と表す.
%\item $\eta_1, \ldots, \eta_k \in V^{*}$について$$\begin{matrix} \eta_1\otimes \cdots \otimes \eta_k  : &V \times \cdots \times V & \rightarrow &\R\\&(v_1, \ldots, v_k)& \mapsto& \eta_1(v_1)\cdots \eta_{k}(v_k)\end{matrix}$$と定義し, $\eta_1\otimes \cdots \otimes \eta_k$を$\eta_1, \ldots, \eta_k$のテンソル積という.
  %\item 特に$\{ e_1, \ldots, e_m\}$を$V$の基底とし, $\{ \omega_1, \ldots, \omega_m\}$は$V^{*}$を $\{ e_1, \ldots, e_m\}$の双対基底とするとき, $\{ \omega_{i_1}\otimes \cdots \otimes  \omega_{i_k}\}_{i_1, \ldots, i_k=1, \ldots,m}$が$\otimes^{k} V^{*}$の基底となる.
  \item $\omega \in \otimes^{k} V^{*}$が$k$次対称形式であるとは, 任意の$k$次の置換$\sigma$と, 任意の$\omega(v_1, \ldots, v_k) \in V^{k}$について$\omega(v_{\sigma(1)}, \ldots, v_{\sigma(k)})=\omega(v_1, \ldots, v_k)$となること
  \item $\omega \in \otimes^{k} V^{*}$が$k$次交代形式であるとは, 任意の$k$次の置換$\sigma$と, 任意の$\omega(v_1, \ldots, v_k) \in V^{k}$について$\omega(v_{\sigma(1)}, \ldots, v_{\sigma(k)})=\sgn(\sigma)\omega(v_1, \ldots, v_k)$となること. $V$の$k$次交代形式の${}_m C_{k}$次元のベクトル空間を$\wedge^{k} V^{*}$で表す. 
\end{itemize}
    \end{dfn}
    \end{tcolorbox}
\begin{exa}
 $\eta_1, \ldots, \eta_k \in V^{*}$について
     $$
      \begin{matrix}
     \eta_1\otimes \cdots \otimes \eta_k  : &V \times \cdots \times V & \rightarrow &\R &
     \eta_1\wedge \cdots \wedge \eta_k  : &V \times \cdots \times V & \rightarrow &\R \\
      &(v_1, \ldots, v_k)& \mapsto& \eta_1(v_1)\cdots \eta_{k}(v_k) & &(v_1, \ldots, v_k)& \mapsto& \det((\eta_i(v_j))_{1 \le i,j \le k})
       \end{matrix}
      $$
      と定義する. $\eta_1\otimes \cdots \otimes \eta_k \in \otimes^{k} V^{*}$であり$\eta_1, \ldots, \eta_k$のテンソル積と呼ばれる. 

$\{ e_1, \ldots, e_m\}$を$V$の基底とし, $\{ \omega_1, \ldots, \omega_m\}$は$V^{*}$を $\{ e_1, \ldots, e_m\}$の双対基底とするとき, $\{ \omega_{i_1}\otimes \cdots \otimes  \omega_{i_k}\}_{i_1, \ldots, i_k=1, \ldots,m}$が$\otimes^{k} V^{*}$の基底となる.また$\{ \omega_{i_1}\wedge \cdots \wedge \omega_{i_k}\}_{1 \le i_1< \cdots < i_k<m}$が$\wedge^{k} V^{*}$の基底となる.
\end{exa}

\item 写像$f : \C^n \rightarrow \C^{n}$を
     $$
      \begin{matrix}
     f : &\C^n & \rightarrow &\C^n\\
      &z=(z_1, \ldots, z_n) & \mapsto& (f_1(z), \ldots, f_n(z))
       \end{matrix}
      $$
%写像$f : \C^n \rightarrow C^{n}$, $f(z_1, \ldots, z_n) = (f_1(z), \ldots, f_n(z))$とする. を各変数について正則であるとする.
とし, 各$f_i(z)$は各変数$z_1, \ldots, z_n$について正則であるとする. 次の問いに答えよ.
\begin{enumerate}
 \setlength{\parskip}{0cm}
  \setlength{\itemsep}{2pt} 
\item $z_i = x_i + \sqrt{-1}y_i$によって$\C^n$に座標$(x_1, y_1, \ldots, x_n, y_n)$を入れる. $f$のヤコビ行列の行列式は常に0以上であることをしめせ.
\item $\C\mathbb{P}^n$は向きづけ可能であることを示せ.\footnote{より一般に複素多様体は向きづけ可能であることがわかる.}
\end{enumerate}

\end{comment}


 \end{document}
